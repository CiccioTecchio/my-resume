\documentclass[%singlesided,
               doublesided,
               paper=a4,
               fontsize=10pt
              ]{my-resume}

\usepackage{hyperref}

\usepackage{xcolor} % Pacchetto per definire i colori personalizzati

% Definizione del colore personalizzato
\definecolor{customblue}{HTML}{007bff}
\usepackage{comment}
\hypersetup{%
     colorlinks=true,
     linkcolor=customblue,
     filecolor=customblue,
     citecolor = black,      
     urlcolor=customblue,
}

%%%%%%%%%%%%%%%%%%%%%%%%%%%%%%%%%%%%%%%%%%%%%%%%%%%%%%%%%%%%%%%%%%%%%%%%%%%%%%%%
% set geometry
%%%%%%%%%%%%%%%%%%%%%%%%%%%%%%%%%%%%%%%%%%%%%%%%%%%%%%%%%%%%%%%%%%%%%%%%%%%%%%%%

\setlength\highlightwidth{8cm}
\setlength\headerheight{3cm}            % note that margintop gets added to this value, i.e. the header bar is 5cm
\setlength\marginleft{1cm}
\setlength\marginright{\marginleft}      % needs to be 1.5 times to be actually equal. why?
\setlength\margintop{1cm}
\setlength\marginbottom{1cm}


%%%%%%%%%%%%%%%%%%%%%%%%%%%%%%%%%%%%%%%%%%%%%%%%%%%%%%%%%%%%%%%%%%%%%%%%%%%%%%%%
% FONTS
%%%%%%%%%%%%%%%%%%%%%%%%%%%%%%%%%%%%%%%%%%%%%%%%%%%%%%%%%%%%%%%%%%%%%%%%%%%%%%%%

\RequirePackage{fontspec}
\setmainfont{Carlito}


%%%%%%%%%%%%%%%%%%%%%%%%%%%%%%%%%%%%%%%%%%%%%%%%%%%%%%%%%%%%%%%%%%%%%%%%%%%%%%%%
% COLORS
%%%%%%%%%%%%%%%%%%%%%%%%%%%%%%%%%%%%%%%%%%%%%%%%%%%%%%%%%%%%%%%%%%%%%%%%%%%%%%%%

\colorlet{highlightbarcolor}{lightgray}
\colorlet{headerbarcolor}{darkgray}

\colorlet{headerfontcolor}{white}
\colorlet{accent}{awesome-red}
\colorlet{heading}{black}
\colorlet{emphasis}{black}
\colorlet{body}{black}


%%%%%%%%%%%%%%%%%%%%%%%%%%%%%%%%%%%%%%%%%%%%%%%%%%%%%%%%%%%%%%%%%%%%%%%%%%%%%%%%
% set document
%%%%%%%%%%%%%%%%%%%%%%%%%%%%%%%%%%%%%%%%%%%%%%%%%%%%%%%%%%%%%%%%%%%%%%%%%%%%%%%%


\begin{document}

\name{Francesco Vicidomini}
\tagline{I am a full-stack developer, when I develop I pay attention to UI/UX and also code reuse.\\
I also really enjoy giving presentations and sharing the results of my work.}
\begin{comment} %old tagline
\tagline{I'm a developer who likes to develop mostly with front-end languages and/or frameworks. I'm working mainly with Angular 2+ and I'm learning React just to change a little bit.\\Currently the topics that interest me most are the Agile methodologies, in particular the Scrum framework, and front-end development trying not to fossilize too much on the code but also trying to take care of the UI and UX of what I develop.}    
\end{comment}

\photo[round]{propic.jpg}{\dimexpr \headerheight-\marginbottom}   % make photo exactly match the header with margintop/marginright/marginbottom as margin

\makeheader

\highlightbar{

    \section{Contact}
    
    \email{fr.vicidomini94@gmail.com}
    \phone{+39 327 5684733}
    \homepage{cicciotecchio.dev}{https://cicciotecchio.dev/}
    \github{@Cicciotecchio}{https://github.com/CiccioTecchio}
    \linkedin{Francesco Vicidomini}{https://www.linkedin.com/in/francesco-vicidomini-374aa813a/}
    \stackoverflow{Francesco Vicidomini}{https://stackoverflow.com/users/4374986/francesco-vicidomini}
    
    \section{Skills}
    
    \skillsection{Programming}
    \skill{OOP}{5}
    \skill{Java}{3}
    \skill{Javascript and Typescript}{4}
    \skill{Angular2+}{4}
    \skill{RxJS}{4}\\
    \skill{NgRx}{4}\\
    \skill{React.js}{3}
    \skill{SQL \& NoSQL DBMS}{3}\\
    \skill{HTML/CSS}{4}\\
    \skill{Node.js}{4}\\
    \skill{Docker}{4}\\
    \skill{Software testing}{3}
    
    
    
    \vspace{0.3em}
    \skillsection{Software \& Tools}
    \skill{Versioning with git}{5}
    \skill{Issue management}{3}\\
    (e.g. trello, gitHub and gitLab issue ecc...)\\
    \skill{Deploy in cloud platform}{3}\\
    (I've deployed some app on SAP CP, AWS EC2 and AWS RDS e my personal Raspberry PI)\\
    \skill{Microsoft Office}{4}
    
    \vspace{0.3em}
    \skillsection{Agile methodology}
    \skill{Scrum}{4}\\
    \skill{Kanban}{3}
    
    \vspace{0.3em}
    \skillsection{Languages}
    \skill{Italian}{5}
    \skill{English}{3}
    
    
    \section{Soft skills}
    \skill{Communication}{4}
    \skill{Self-organization}{3}
    \skill{Proactivity}{4}
    
    %da sbloccare al più presto
    % \section{Certifications}
    % \simpleskill{PSM-1}
    
    \section{Personal hobbies}
    \simpleskill{Workout}
    \simpleskill{Ride my motorcycle}
    \simpleskill{Read books}
    \simpleskill{Listen to music}
    \simpleskill{Play videogames}
    \simpleskill{Watch movies or TV series}
    

}
\mainbar{
    \section[\faGears]{Work history}
    \job{07/2023 - Current}
        {Engineering, Naples, Italy}
        {Software Development Specialist}
        {In Engineering, I am involved in various aspects of SW development, not just the front-end. I also deal with back-end development and continuous deployment aspects. The projects I am working on are mainly in the area of cybersecurity.}
    \job{06/2020 - 07/2023}
        {NTT Data, Naples, Italy}
        {Software engineer}
        {I work in an Agile environment on the development, testing, and client release of web interfaces designed in Angular2+.}
    \job{05/2015 - 08/2016}
        {Inventa CPM, Nocera Superiore, Italy}
        {Sales promoter}
        {During this activity I was in charge of the promotion and sale of Samsung mobile products, the promotion and sales activity was carried out at the Shopping Center Nuceria Nocera Superiore (SA).}
    
    \section[\faMortarBoard]{Education}
    \job{01/2018 - 07/2020}
        {Università degli Studi di Salerno}
        {Master Degree in Computer Science}
        {Curricula of Software Engineering, Final grade 110L}
    
    \job{09/2013 - 12/2017}
        {Università degli Studi di Salerno}
        {Bachelor Degree in Computer Science}
        {Final grade 94}

    \section[\faCode]{Projects}
        \job{07/2023 - Current}
        {Engineering}
        {\href{https://encrypt-project.eu/}{ENCRYPT - Project}}
        {This project allowed me to deepen my knowledge of computer security and personal data processing by developing a web application that allows users to carry out a privacy risk assessment.\\
        In addition to developing the web application, I was also responsible for creating presentations and articles that could be used to present the project at meetings and conferences.\\ \textbf{Technology stack}: Angular 18, RxJS, NgRx, Node.js, MongoDB, Docker.}
        \job{06/2020 - 07/2023}
        {NTT Data}
        {Pab goes digital}
        {This project allows for the digitization of administrative procedures that are currently handled through paper applications and face-to-face interactions with the public offices of the Province of Bolzano.\\ \textbf{Technology stack}: Angular 8, Spring, MySQL, SAP CP.}
    \job{Current}
        {Myself}
        {\href{https://github.com/CiccioTecchio/CiccioTecchio.dev}{Cicciotecchio.dev}}
        {My personal website.\textbf{Technology stack}: React, ReactBootstrap, GhPages.}
    \job{09/2018 - 02/2019}
        {Università degli Studi di Salerno}
        {\href{https://github.com/CiccioTecchio/SharErasmus}{ShareErasmus}}
        {I have carried out the activity of project manager coordinating 8 students in order to create a support platform for Erasmus students.\\\textbf{Technology stack}: JQuery, Node.js, MySQL, Firebase, AWS EC2, AWS RDS.}
        
    \job{02/2019 - 11/2019}
        {Università degli Studi di Salerno}
        {\href{https://github.com/CiccioTecchio/YASPL3}{YASPL3}}
        {This is a simple programming language that I've implemented for the exam of Compilatori.\\ \textbf{Technology stack}: Java, LeX Java CUP.}
    
    \job{06/2018 - 06/2018}
        {Università degli Studi di Salerno}
        {\href{https://github.com/CiccioTecchio/n-Body_MPI}{N-body}}
        {I paralleled the n-body simulation algorithm using OpenMp and tested the results on an AWS cluster.\\ \textbf{Technology stack}: C, Open Mp, AWS EC2.}
        
    
    %\section{Achievements, honours and awards}
    %\achievement{My first achievement}
    %\achievement{My second achievement}


    %\section{General Skills}
    %\smallskip % additional skip because tag outlines use up space
    %\tag{Tag 1}
    %\tag{Tag 2}
    %\tag{and}
    %\tag{another tag}
    %\tag{some more tags}
    %\tag{yet another one}
    %\tag{tags flow over}
    %\tag{to the next line}
    %\tag{if necessary}
    
    %\medskip
    %Tags must be ordered by hand with newlines to get a nice layout, especially for long tags.
    
    %\section{Wheel Chart}
    % This is taken from AltaCV
    % see https://github.com/liantze/AltaCV for details
    %\wheelchart{1.5cm}{0.5cm}{% outer and inner diameter
    %    6/8em/accent!20/Sleep,          % comma-separated list of
    %    8/8em/accent!40/Daytime job,    % fraction of 24 / line length / color / label
    %   2/8em/accent!80/Training,          % here, the color is shades of the accent color
    %    3/8em/accent!60/Recovering from fighting criminals,
    %   5/8em/accent/Being Batman
    %}
}

\makebody


\begin{comment}

\pagestyle{highlightmain}

% The highlightbar needs to be filled to display mainbar contents correctly in singlesised mode
% For an empty highlightbar, fill with empty space
\highlightbar{\hfill}
\mainbar{

    \section{Another section}
    
    This page uses the page style \texttt{highlightmain} which shows the highlight bar (gray) and the main part (white background) but omits the header. 
    The default page style is \texttt{headerhighlightmain} with all three elements.
    If you don't want header, nor highlight bar, use page style \texttt{\textbackslash pagestyle\{empty\}}.
    \medskip
    Neither main, nor highlight bar must be filled to make this template work.
    It is possible to use a page style with the highlight bar but leave it empty by setting an empty highlightbar \texttt{\textbackslash highlightbar\{\}}.

    \vspace{0.5em}
    \subsection{Subsection 1}
    Demonstrate subsections.
    
    \subsection{Subsection 2}
    Subsection are also bold face but a smaller font then section. They also omit the rule.
    

}

\makebody
\end{comment}

\clearpage
\pagestyle{empty}

\section{Publications}
\pubforcefullwidth
\publication{Implementation of a tool for risk assessment of personal data processing using a methodology based on the LINDDUN and MITRE ATT\&CK frameworks}
{Francsco Vicidomini, Paolo Roccetti, Vincenzo Napolitano}
{2025}
{\href{https://encrypt-project.eu/communication/encrypt-blog/}{Posted on ENCRYPT Project blog}}
{}
\publication
	{Ex Machina: Analytical platforms, Law and the Challenges of Computational Legal Science} % Title
	{Nicola Lettieri, Antonio Altamura, Rosalba Giugno, Alfonso Guarino, Delfina Malandrino, Alfredo Pulvirenti, Francesco Vicidomini, and Rocco Zaccagnino} % Authors
	{2018} % Year
	{\href{https://www.mdpi.com/1999-5903/10/5/37}{Future Internet 2018, 10(5), 37}} % Journal
	{} % ADS & arxiv links
\section{TREATMENT OF PERSONAL DATA}
According to law 679/2016 of the Regulation of the European Parliament of 27th April 2016, I hereby express my consent to process and use my data provided in this CV and application for recruiting purposes.

\end{document}